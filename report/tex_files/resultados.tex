\section{Resultados}
Se ha explorado distintas técnicas de biología de sistemas con el objetivo de entender el fenotipo \textit{Frontotemporal Dementia} y sus relaciones con genes y las funciones de estas. 
Mediante el análisis de redes de interacción proteína-proteína, de los genes relacionados con el fenotipo, obtenidos mediante StringDB.

\subsection{Red PPI y sus propiedades}

HPO permite extraer los genes anotados a un fenotipo, por lo que estos han sido descargados y utilizando la API de StringDB se han obtenido la red de interacción (\textit{Figura} \ref{fig:network}).
Se ha establecido el umbral de \textit{combine\_score} a 700.

\begin{figure}[h]
	\centering
	\includegraphics[width=1\linewidth]{../results/plots/network/network.png}
	\caption{Red de interacción proteína-proteína de los genes del HPO (HP:0002145) obtenida mediante la API de StringDB con un filtro de 400 del combine score}
	\label{fig:network}
\end{figure}

En la \textit{Figura \ref{fig:network}} observamos que proteínas con un número de conexiones respecto a otras. 
Un análisis en mayor profundidad de la red se presenta en la \textit{Tabla \ref{tab:NetworkMetrics}}.


\begin{table}[h!]
	\centering
	\caption{Resumen de Métricas de la Red PPI}
	\label{tab:NetworkMetrics}
	\begin{tabular}{ll}
		\toprule
		\textbf{Categoría} & \textbf{Métrica y Valor} \\
		\midrule
		\multirow{2}{*}{\textbf{Tamaño de la Red}} 
		& Número de nodos: 42 \\
		& Número de aristas: 171 \\
		\midrule
		\multirow{2}{*}{\textbf{Grado}} 
		& Grado promedio: 8.14 \\
		& Desviación estándar del grado: 5.93 \\
		\midrule
		\multirow{3}{*}{\textbf{Conectividad}} 
		& Grafo conectado: Sí \\
		& Conectividad de nodos: 1 \\
		& Conectividad de aristas: 1 \\
		\midrule
		\multirow{2}{*}{\textbf{Densidad y Sparsity}} 
		& Densidad: 0.199 \\
		& Esparcidad: 0.801 \\
		\midrule
		\multirow{2}{*}{\textbf{Cercanía (Closeness)}} 
		& Cercanía promedio: 0.456 \\
		& Desviación estándar de cercanía: 0.093 \\
		\midrule
		\multirow{2}{*}{\textbf{Centralidad (Betweenness)}} 
		& Betweenness promedio: 26.50 \\
		& Desviación estándar de betweenness: 39.56 \\
		\midrule
		\multirow{3}{*}{\textbf{Transitividad}} 
		& Transitividad local promedio: 0.550 \\
		& Desviación estándar de transitividad local: 0.322 \\
		& Transitividad global: 0.520 \\
		\midrule
		\multirow{3}{*}{\textbf{Otras Propiedades}} 
		& Asortatividad: -0.018 \\
		& Diámetro: 6 \\
		& Longitud de camino promedio: 2.293 \\
		\bottomrule
	\end{tabular}
\end{table}

En HPO se tenían 52 genes relacionados con nuestro fenotipo, pero en la red filtrado tan solo hay 42 proteínas. 
Por tanto hay diez genes, que no interaccionan con las otras proteínas de la red con un alto umbral de confianza. 
Con un umbral inferior al establecido,  estos genes estarían en la red, pero la significancia biología de las interacciones no sería fiable.

El grado medio de los nodos, correspondiente al número de conexiones de un nodo, es 8.14, un valor considerablemente alto si tenemos en cuenta el número de nodos presentes en la red. En total, la red tiene 171 conexiones ente proteínas.
Hay un valor significativo de desviación típica del grado de los nodos. Un mejor análisis de las distribución de los grados se presenta en la \textit{Figura \ref{fig:degree}}. 
En esta figura, observamos que la gran mayoría de las proteínas, tienen grados relativamente bajos, inferiores a la medio. Encontramos un par de proteínas, con grados superiores a 20, lo que indica que están conectados a más de la mitad de las proteínas de la red.
Estos nodos altamente conectados podrían ser hubs (FUS, TARDBP). %cre que no lo son (mirar conectividad y centralidad) %aunque si los quitas el grafo sigue conexo, mirar en que pathways están incluidos
Si nos fijamos en la densidad del grafo, vemos que tenemos una red con baja densidad, hay pocas conexiones con respecto a las posibles.


\begin{figure}[h]
	\centering
	\includegraphics[width=1\linewidth]{../results/plots/network/degree_distribution.pdf}
	\caption{Distribución de los grados de los nodos, mediante un histograma, de la red de interacción.}
	\label{fig:degree}
\end{figure}

La red obtenida es un grafo conexo (ver \textit{Figura \ref{fig:network}}), el análisis de la conectivadad llevado a cabo muestra que existe una conectividad tanto de vértice (nodo) como de arista de 1.
Esto indica que exiten uno o varios nodos/aristas que si son eliminados desconectan el grafo. Se ha determinado que estos nodos son las proteínas NEK1, SOD1, APP, PON1, NEFH Y PSEN1.

Las siguientes métrica analizadas son la centraliad y cercanía, en la \textit{Tabla \ref{tab:NetworkMetrics}} se muestra la media y desviación, 
ya que se tratan de métricas que se miden por cada nodo. Para ambas métricas destaca una de las dos proteínas antes mencionadas con el grado,TARDBP.
%añadir más explicación

\begin{figure}[h]
	\centering
	\includegraphics[width=1\linewidth]{../results/plots/network/closeness_betwennes_graph.pdf}
	\caption{Representación visual de la relación entre closeness (cercanía) y betweenness (centralidad de intermediación) para los nodos de la red. 
    El tamaño de los nodos ha sido establecido en base al valor de centralidad obtenido y el valor de cercanía se establece con el color del nodo.}
	\label{fig:closeness_betweenness}
\end{figure}

Finalmente podemos mencionar la transitividad, probabilidad de que los vecinos de un nodo estén también conectados entre sí.
Tanto la transitividad local como la global tienen valores parecidos, entorno a 0.5. Puntualizar que para el calculo de la transitividad local, se han obviado aquellos nodos que dan valores de NaN debido a que solo presentan un vecino.



A lo largo de esta sección se presentan los resultados de los procedimientos descritos en la metodología. 

\subsection{Análisis de la Red}

\subsection{Optimización de Hiperparámetros}

El proceso de optimización de hiperparámetros mediante el BHO se analizará mostrando el frente de pareto de ambos algoritmos y una visualización del rendimiento marginal de ambas métricas en base al valor de resolución (\(\gamma\)) evaluado.

\begin{figure}[htbp]
	\centering
	\includegraphics[width=.96\textwidth]{../results/plots/optimization/pareto_comparison.pdf}
	\caption{Frente de Pareto obtenido tras maximizar el puntuaje de FES y el coeficiente Q para Leiden y Louvain. Se puede observar que, para las soluciones Pareto-eficientes que conforman el extremo de la frontera, Leiden obtiene un puntuaje superior en cuanto a la modularidad, mientras que ambos algoritmos están igualados en cuanto al puntuaje de enriquecimiento funcional, diferenciándose por pocos decimales.}
	\label{fig:pareto}
\end{figure}

\begin{figure}[htbp]
	\centering
	\includegraphics[width=.96\textwidth]{../results/plots/optimization/hyperparameter_vs_metric.pdf}
	\caption{Valores de resolución (\(\gamma\)) explorados por algoritmo, frente al rendimiento evaluado sobre el clustering obtenido. Leiden obtiene por un margen considerable la mejor modularidad, mientras que Louvain obtiene el mejor valor del FES por una diferencia despreciable.}
	\label{fig:slice_plot}
\end{figure}


\subsection{Clustering}

\begin{figure}[htbp]
	\centering
	\includegraphics[width=.75\textwidth]{../results/plots/clustering/fast greedy_clustering.pdf}
	\caption{Resultados de clustering para el algoritmo Fast-Greedy. Este clustering será considerado como nuestra base. No se ha ajustado ningún hiperparámetro para este método.}
	\label{fig:fastgreedy_clustering}
\end{figure}

\begin{figure}[htbp]
	\centering
	\includegraphics[width=.96\textwidth]{../results/plots/clustering/leiden_clustering.pdf}
	\caption{Resultados de clustering para el algoritmo Leiden. Se presentan dos resultados, que corresponden a las soluciones Pareto-óptimas extremas (\(max(Q)\) ó \(max(FES)\)).}
	\label{fig:leiden_clustering}
\end{figure}

\begin{figure}[htbp]
	\centering
	\includegraphics[width=.96\textwidth]{../results/plots/clustering/louvain_clustering.pdf}
	\caption{Resultados de clustering para el algoritmo Louvain. Se presentan dos resultados, que corresponden a las soluciones Pareto-óptimas extremas (\(max(Q)\) ó \(max(FES)\)).}
	\label{fig:louvain_clustering}
\end{figure}

\subsection{Análisis Funcional}


\begin{figure}[htbp]
	\centering
	\includegraphics[width=.75\textwidth]{../results/plots/functional_analysis/venn_plot.pdf}
	\caption{Este diagrama de Venn compara los términos funcionales identificados al aplicar el algoritmo de Leiden utilizando dos criterios: máxima modularidad y máximo puntaje de enriquecimiento funcional. La superposición indica los términos compartidos entre ambas soluciones, lo que proporciona una visión de la robustez y consistencia de los procesos identificados. Esta comparación ayuda a validar los resultados del análisis de clustering y enriquecimiento funcional​}
	\label{fig:venn_plot}
\end{figure}


\begin{figure}[htbp]
	\centering
	\includegraphics[width=.75\textwidth]{../results/plots/functional_analysis/bar_plot.pdf}
	\caption{Este gráfico de barras muestra aquellos procesos biológicos enriquecidos del fenotipo de demencia frontotemporal (DFT). Los términos GO (Gene Ontology) más relevantes están ordenados por el p-valor ajustado en una escala logarítmica. La longitud de las barras indica la magnitud del enriquecimiento, mientras que el p-valor ajustado refleja la significancia estadística de cada proceso identificado​.}
	\label{fig:bar_plot}
\end{figure}

\begin{figure}[htbp]
	\centering
	\includegraphics[width=.75\textwidth]{../results/plots/functional_analysis/dot_plot.pdf}
	\caption{Este gráfico de puntos visualiza los términos GO enriquecidos junto con el Gene Ratio y el p-valor ajustado. El Gene Ratio indica la proporción de genes asociados a cada proceso respecto al total de genes en la lista de entrada. El tamaño del punto representa el número de genes implicados en cada término, mientras que el color refleja el p-valor ajustado.}
	\label{fig:dot_plot}
\end{figure}

\begin{figure}[htbp]
	\centering
	\includegraphics[width=.75\textwidth]{../results/plots/functional_analysis/cnet_plot.pdf}
	\caption{Este gráfico muestra la relación entre genes específicos y los procesos biológicos enriquecidos en la DFT. Cada término GO está conectado con los genes que participan en dicho proceso. Esta visualización permite identificar genes que actúan como puntos de convergencia funcional.}
	\label{fig:cnet_plot}
\end{figure}

