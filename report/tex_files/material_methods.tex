\section{Materiales y métodos}

\subsection{Datos}

\vspace{0,3cm}

\subsubsection*{Datos del fenotipo}

Para este estudio, se utilizó el fenotipo Demencia frontotemporal, identificado con el término HP:0002145 en Human Phenotype Ontology (HPO). A partir de este término HPO, se han extraído 52 genes asociados al fenotipo. Estos genes se obtuvieron mediante la API Ontology Annotation Network \cite{hpo_api} de HPO, que permite acceder programáticamente a las anotaciones entre términos fenotípicos y genes. A través de esta API, se descargaron los datos en formato JSON, que luego fueron procesados para extraer los nombres de los genes y guardarlos en un archivo TSV.

Estos genes, relacionados con el desarrollo de la demencia frontotemporal y otras patologías neurodegenerativas, representan el conjunto inicial de genes sobre el que se construirá la red de interacciones para el análisis posterior. Cada uno de ellos se identifica mediante su ID único en la base de datos de NCBIGene, lo cual facilita el acceso y la referencia a los datos genéticos específicos.

Para asegurar la reproducibilidad, se utilizó la versión 2.0.4 de HPO  \cite{HPO}, para obtener el término fenotípico y descargar los genes relacionados con el fenotipo de estudio. En la siguiente sección se proporciona una descripción detallada de HPO.
 
\subsubsection*{Human Phenotype Ontology (HPO)}

HPO proporciona una ontología estandarizada que describe anomalías fenotípicas observadas en enfermedades humanas, facilitando la identificación y análisis de genes asociados a diversas características clínicas. Cada término en HPO representa una anomalía específica, como la demencia frontotemporal, y está diseñado para facilitar la caracterización precisa de los fenotipos en el contexto de enfermedades hereditarias. La ontología se desarrolla y actualiza de forma continua utilizando fuentes como la literatura médica, así como bases de datos como Orphanet, DECIPHER y OMIM. Actualmente, HPO contiene más de 18,000 términos y ofrece más de 156,000 anotaciones asociadas a enfermedades hereditarias \cite{HPO}.


\subsubsection*{Datos de interacción}

Los datos de interacción representan conexiones funcionales y físicas entre proteínas, y constituyen la base para construir redes de interacción en el análisis de procesos biológicos. En este estudio, los datos de interacción proteína-proteína (PPI) fueron extraídos de la base de datos STRING mediante su API REST \cite{string_api}, que permite recuperar programáticamente redes de interacción específicas basadas en listas de genes o proteínas de interés.

A través de esta API, se obtuvieron las interacciones entre los genes asociados al fenotipo FTD (HP:0002145) en formato TSV. En este archivo, cada fila representa una interacción entre dos proteínas y contiene las siguientes columnas.

\begin{itemize}
	\item \textbf{protein1:} ID de la primera proteína en la interacción, precedido por el código taxonómico del organismo (por ejemplo, "9606" para proteínas humanas).
	\item \textbf{protein2:} ID de la segunda proteína en la interacción, también con el prefijo de organismo.
	\item \textbf{combined\_score:} Puntuación de confianza combinada para cada interacción proteína-proteína, con valores que oscilan entre 0 y 1000. Esta puntuación refleja la probabilidad de que una interacción sea real, basada en una integración de diversas fuentes de evidencia, como co-ocurrencia filogenética, co-expresión, minería de texto y datos experimentales. Cada tipo de evidencia se evalúa y puntúa individualmente, y luego se combina en el "combined\_score", proporcionando así un indicador global de confiabilidad para cada interacción funcional o física \cite{szklarczyk2023stringdb}.
\end{itemize}

Estos datos obtenidos se utilizarán para construir una red de interacciones entre los genes asociados al fenotipo FTD, permitiendo analizar las relaciones funcionales entre proteínas en este contexto. Esta red servirá como base para el análisis de clustering, facilitando la identificación de módulos de genes potencialmente implicados en funciones biológicas específicas.

Para asegurar la reproducibilidad del análisis, se utilizó la versión 12.0 de STRING, junto con su API REST de STRING para la extracción de interacciones.


\subsubsection*{STRING}
La base de datos STRING (Search Tool for the Retrieval of Interacting Genes/Proteins) es un recurso bioinformático diseñado para recopilar, organizar y analizar redes de interacciones proteína-proteína y asociaciones funcionales en cualquier genoma secuenciado. STRING integra información de diversas fuentes, como minería de texto científico, predicciones computacionales basadas en coexpresión y contexto genómico, y datos experimentales obtenidos de estudios de interacciones proteicas. Además, los usuarios pueden acceder a la base de datos para explorar redes de interacción, realizar análisis de enriquecimiento funcional y generar redes personalizadas para genomas específicos, facilitando así la investigación en biología celular y molecular. Actualmente, STRING cubre 59.309.604 proteínas provenientes de 12.535 organismos.  \cite{szklarczyk2023stringdb}.

\subsection{Software}

Para el análisis funcional y la construcción de redes genéticas en este estudio, se seleccionaron herramientas especializadas que permiten tanto la exploración bioinformática como la visualización de datos complejos. Dado que el objetivo principal es investigar la interacción entre genes y módulos específicos asociados a la demencia frontotemporal, se ha optado por una combinación de paquetes en Python y R que ofrecen un balance entre precisión analítica y capacidades visuales avanzadas.

\subsection*{Paquetes de Python para el análisis funcional y otras funciones}

\begin{itemize}
	\item \textbf{Pandas (versión 2.2.3)}: Este paquete proporciona estructuras de datos eficientes y flexibles, como DataFrames, que facilitan el procesamiento y manipulación de datos complejos. En el contexto de este estudio, Pandas permite organizar, filtrar y procesar resultados de enriquecimiento funcional, simplificando el manejo de grandes volúmenes de datos bioinformáticos \cite{pandas}.
	\item \textbf{Matplotlib (versión 3.8.1)}: Matplotlib es una librería de visualización muy versátil que soporta múltiples tipos de gráficos en 2D, lo que resulta útil para representar tendencias y relaciones entre genes en gráficos de líneas, barras, dispersión, y más. Este paquete se utilizará para visualizar los resultados de enriquecimiento y las interacciones génicas \cite{matplotlib}.
	\item \textbf{Scienceplots (versión 2.1.1)}: Este paquete extiende Matplotlib proporcionando estilos de gráficos estéticamente optimizados para publicaciones científicas. Con Scienceplots, se puede lograr una presentación visual de alta calidad, ideal para gráficos que requieren una apariencia profesional \cite{scienceplots}.
	\item \textbf{Requests (versión 2.31.0)}: Requests es una biblioteca para realizar solicitudes HTTP de manera simple y efectiva. En este estudio, se utiliza para conectar con APIs externas como la de STRINGdb, permitiendo la descarga automatizada de datos de redes y enriquecimiento \cite{requests}.
	\item \textbf{GOATOOLS (versión 1.2.3)}: Este paquete permite realizar análisis de enriquecimiento funcional en términos de Gene Ontology (GO), lo que facilita la identificación de términos GO sobre-representados en conjuntos de genes. Es una herramienta clave para comprender funciones biológicas asociadas a los genes estudiados \cite{goatools}.
	\item \textbf{G:Profiler (versión 1.4.0)}: G:Profiler realiza análisis de enriquecimiento funcional abarcando varias bases de datos (GO, KEGG, Reactome, entre otras). La versatilidad de este paquete lo convierte en una excelente opción para explorar las funciones biológicas de conjuntos de genes con una amplia cobertura de recursos \cite{gprofiler}.
	\item \textbf{Statsmodels (versión 0.14.0)}: Statsmodels es una biblioteca de estadística que permite aplicar ajustes de p-valor, como el método de Benjamini-Hochberg, para reducir el impacto de falsos positivos en los resultados de enriquecimiento. Esto asegura una interpretación estadísticamente robusta de los resultados \cite{statsmodels}.
\end{itemize}

\subsection*{Paquetes de R para clustering y visualización de redes}

\begin{itemize}
	\item \textbf{Cluster (versión 2.1.5)}: Este paquete ofrece una variedad de algoritmos clásicos de clustering (como k-means y clustering jerárquico), facilitando la agrupación de genes según patrones de expresión similares, lo cual es fundamental para identificar posibles módulos relacionados con la demencia frontotemporal \cite{cluster}.
	\item \textbf{Factoextra (versión 1.0.8)}: Factoextra permite una visualización clara y accesible de resultados de clustering, proporcionando gráficos intuitivos para explorar la estructura de los datos y resaltar posibles módulos o patrones de coexpresión \cite{factoextra}.
	\item \textbf{WGCNA (versión 1.71)}: Este paquete es ideal para detectar módulos de genes coexpresados mediante la construcción de redes de coexpresión genética ponderada, lo cual es útil en estudios de expresión génica complejos como el de la demencia frontotemporal \cite{wgcna}.
	\item \textbf{Igraph (versión 1.5.0)}: Igraph facilita la construcción, manipulación y visualización de redes genéticas y módulos de coexpresión. Este paquete es útil para analizar la conectividad y relación entre genes en un contexto de red \cite{igraph}.
	\item \textbf{STRINGdb (versión 2.10.0)}: STRINGdb es un paquete que conecta con la base de datos STRING, permitiendo realizar enriquecimiento funcional en redes y visualizar interacciones proteicas y genéticas. Es útil para identificar las relaciones funcionales entre genes \cite{stringdb}.
	\item \textbf{Pathview (versión 1.36.0)}: Este paquete permite integrar datos de expresión génica en rutas de KEGG, ofreciendo representaciones gráficas que muestran la implicación de los genes en rutas metabólicas y de señalización, lo cual es relevante para entender su rol en la enfermedad \cite{pathview}.
	\item \textbf{ClusterProfiler (versión 4.15.0)}: ClusterProfiler permite realizar análisis de enriquecimiento funcional de alta precisión en términos de GO, KEGG y Reactome. Ofrece opciones avanzadas de visualización y ajuste de p-valores, facilitando la interpretación de los resultados \cite{clusterprofiler}.
\end{itemize}

\subsection{Análisis de enriquecimiento de vías biológicas}

El análisis de enriquecimiento permite identificar vías biológicas o pathways que están significativamente representados en una lista de genes de interés, por medio de prubas estadísticas. Un pathway es un conjunto de genes que trabajan en conjunto para llevar a cabo un proceso biológico específico.\cite{Reimand2019}

Se realizó un análisis de enriquecimiento mediante la API de STRINGdb. Mediante el mapeo de las anotaciones funcionales de los genes en ontologías como Gene Ontology (GO), UniProt, Reactome y KEGG y comparando la distribución de estos términos en el conjunto de genes de interés con su distribución en un conjunto de referencia, se identificaron estadísticamente los términos sobrerrepresentados en la lista de genes asociados a la enfermedad. \cite{Tipney2010}

STRINGdb, por tanto, permite analizar diversas categorías de anotaciones funcionales, incluyendo los términos de Gene Ontology (procesos biológicos, funciones moleculares y componentes celulares), rutas de KEGG y Reactome \cite{szklarczyk2023stringdb}. Se realizó un filtrado según el tipo de término para obtener los diferentes enriquecimientos funcionales de interés.

Para calcular el enriquecimiento, STRINGdb utiliza un test hipergeométrico que evalúa si la representación de cada término funcional en el conjunto de genes de interés es mayor que la esperada por azar. Para controlar el error debido a comparaciones múltiples, se aplica la corrección de Benjamini-Hochberg para ajustar la tasa de descubrimiento falso (FDR) \cite{szklarczyk2023stringdb}.

El análisis de enriquecimiento se realizó tanto en el conjunto completo de genes asociados a la enfermedad, extraído de términos de HPO, como en los genes agrupados en clústeres. 


\subsection{Clustering}

Al aplicar algoritmos de clustering, nuestro objetivo es descubrir comunidades funcionales dentro de la red de genes asociada a la demencia frontotemporal. Estas comunidades, módulos, o \textit{clusters} funcionales pueden representar procesos biológicos específicos, vías celulares, o mecanismos asociados al fenotipo FTD. Encontrar estos clusters podría revelar posibles dianas terapéuticas o grupos de biomarcadores dentro del conjunto de genes, lo cual podría abrir las puertas a nuevos tratamientos para los pacientes de FTD.

\subsubsection{Red}

En la red de interacción proteína-proteína (PPI) obtenida mediante la API de STRINGdb, cada nodo representa un gen y cada arista representa una interacción entre genes, ponderada por un puntaje de confianza otorgado por STRINGdb \cite{szklarczyk2023stringdb}. Para enfocarnos en interacciones más sólidas y confiables, se aplicó un umbral mínimo de confianza en las aristas, lo que permitió refinar la red y mejorar la relevancia de los clústeres detectados 

% (\textbf{NOTA:} Se debe ahondar en esto si al final se hace: ¿Por qué ese umbral? ¿Y si fueran otros umbrales?).

\input{tex_files/medidas_rendimiento.tex}
\subsubsection{Algoritmos}
\label{sec:algoritmos}

A continuación, se detallan los tres algoritmos de clustering, proporcionados por la librería \textit{iGraph}, elegidos para este estudio, los cuales pretenden cubrir diferentes enfoques teóricos en la detección de comunidades funcionales \cite{igraph}.  

\begin{itemize}
    \item \textbf{Fast Greedy}: es un algoritmo de clustering jerárquico que optimiza directamente la modularidad, lo que le confiere una gran utilidad en biología de sistemas, ya que la esta captura la idea de que los nodos dentro de una comunidad son más conexos entre sí que con nodos de otras comunidades.
    
    La estrategia que sigue Fast Greedy es voraz, es decir que toma decisiones locales en cada iteración para optimizar la modularidad. Sigue los siguientes pasos: inicialización (considera que cada nodo es un cluster), fusión (en cada iteración fusiona las comunidades que aumenten en mayor medida la modularidad) y terminación (el algoritmo termina cuando no se pueda incrementar más la modularidad) \cite{clauset2004finding}.

    Este algoritmo no precisa del ajuste de ningún hiperparámetro, por lo que los resultados del mismo se han tomado como referencia y punto de partida para los demás algoritmos.
    
    % (\textbf{NOTA:} Hay un parámetro intersante, \textit{initial\_membership} el cual son nodos 'semilla' que se pasan como argumento, y el algoritmo intenta mejorar las comunidades alrededor de estos nodos. Podríamos usar genes del análisis funcional como semilla y ver qué pasa.)
    
    \item \textbf{Algoritmo Louvain}: este algoritmo es uno de los más utilizados para la detección de comunidades en redes. Al igual que el algoritmo Fast Greedy, se basa en optimizar la modularidad de manera jerárquica. Tiene dos fases claves en su funcionamiento:

    \begin{itemize}
        \item \textit{Optimización local de modularidad}: al inicio, se asigna a cada nodo su propio cluster. En cada iteración, se evalúa si mover un nodo a la comunidad de uno de sus vecinos incrementa la modularidad de la red. El nodo se mueve a la comunidad que maximiza la modularidad local.

        \item \textit{Construcción de la red}: una vez los nodos están en comunidades correspondientes, se agrupan las comunidades en un nuevo “supernodo” y se construye una nueva red en la que los nodos son las comunidades encontradas. Se vuelve a calcular la modularidad y se repite el proceso hasta que no se pueda mejorar más la modularidad.
    \end{itemize}

    Este proceso jerárquico permite detectar comunidades a diferentes escalas de la red \cite{Blondel2008Louvain}.

    Se ajustó el parámetro de resolución, que controla el tamaño final de las comunidades. El resto de parámetros se dejaron con sus valores por defecto.
    
    \item \textbf{Algoritmo de Leiden}: este algoritmo se diseñó para mejorar las limitaciones del algoritmo de Louvain, particularmente en términos de garantizar comunidades bien conectadas. Sigue los siguintes pasos:

    \begin{itemize}
        \item \textit{Movimiento local de nodos}: Cada nodo del grafo comienza en su propia comunidad. El algoritmo evalúa si mover un nodo a la comunidad de uno de sus vecinos incrementa la modularidad. Si es así, el nodo se mueve. Este proceso se repite hasta que ningún movimiento adicional mejore la modularidad.

        \item \textit{División interna de comunidades}: Dentro de cada comunidad, el algoritmo verifica si estas son completamente conexas. Si no lo son, divide las comunidades en subcomunidades más pequeñas para garantizar que todas sean subgrafos conexos.

        \item \textit{Agregación y simplificación del grafo}: Cada comunidad identificada se trata como un solo nodo, y se construye un nuevo grafo 'resumido'. Luego, se repiten los pasos anteriores con este nuevo grafo \cite{traag2019leiden}.
    \end{itemize}
    
    Se ajustó el parámetro \(\gamma\), así como el número de iteraciones del algoritmo, permitiendo que el Leiden refinara iterativamente la partición. El resto de parámetros se fijaron a su valor por defecto. 

\end{itemize}


\subsubsection{Optimización}

Como se explicó en la Sección \ref{sec:algoritmos}, Louvain y Leiden son algoritmos de clustering con múltiples parámetros configurables. Ajustar adecuadamente estos parámetros es crucial para mejorar tanto la interpretación biológica de la red como la detección de comunidades. Este proceso, conocido como ajuste de hiperparámetros, busca optimizar el rendimiento del algoritmo según diversas métricas. En este apartado, se detalla el procedimiento empleado para ajustar el parámetro de resolución (\(\gamma\)) en Leiden y Louvain, con el objetivo de maximizar las métricas descritas en la Sección \ref{sec:metricas}.

La estadística bayesiana es un enfoque probabilístico que utiliza el teorema de Bayes para actualizar las creencias sobre un modelo a medida que se incorporan nuevos datos. El ajuste bayesiano de hiperparámetros (BHO) aplica este enfoque para optimizar los parámetros de un modelo, construyendo y actualizando iterativamente un modelo probabilístico de la función objetivo en función de los hiperparámetros. Este modelo probabilístico sugiere los hiperparámetros a probar en cada iteración, permitiendo enfocar eficientemente la búsqueda en las regiones más prometedoras del espacio de hiperparámetros.

Existen diversos algoritmos basados en BHO, cada uno caracterizado por el modelo probabilístico que construye. En este proyecto, hemos optado por el Tree-structured Parzen Estimator (TPE). En el TPE, consideramos que \( y \) es el valor obtenido al evaluar la función objetivo utilizando un conjunto de hiperparámetros \( \theta \). Definimos un umbral \( y^* \) que nos permite dividir nuestros datos en dos conjuntos: los hiperparámetros que resultan en un rendimiento mejor que \( y^* \) y los que resultan en un rendimiento peor. Es decir:

\begin{equation}
	\begin{aligned}
		\mathcal{C}_1 = \left\{ \theta^{(i)} \mid y^{(i)} \leq y^* \right\} \\
		\mathcal{C}_2 = \left\{ \theta^{(i)} \mid y^{(i)} > y^* \right\}
	\end{aligned}
\end{equation}

Utilizando estimación de densidad por kernel (KDE), en este caso, el método de ventana deslizante de Parzen-Rosenblatt -ver detalles en Sección \ref{sec:anexo_a}- estimamos las distribuciones de probabilidad para estos dos conjuntos:

\begin{equation}
	\begin{aligned}
		l(\theta) &= p(\theta \mid \theta \in \mathcal{C}_1) \\
		g(\theta) &= p(\theta \mid \theta \in \mathcal{C}_2)
	\end{aligned}
\end{equation}

El objetivo es seleccionar nuevos hiperparámetros \( \theta \) que maximicen el Expected Improvement (EI), que es proporcional al cociente de estas dos densidades:

\begin{equation}
	\mathrm{EI}(\theta) \propto \frac{l(\theta)}{g(\theta)}
\end{equation}

Al maximizar \( \mathrm{EI}(\theta) \), favorecemos los hiperparámetros que son más probables en el conjunto de buen rendimiento \( \mathcal{C}_1 \) y menos probables en el conjunto de peor rendimiento \( \mathcal{C}_2 \). Este proceso se repite iterativamente; en cada iteración, los nuevos datos de rendimiento actualizan \( l(\theta) \) y \( g(\theta) \), permitiendo explorar eficientemente el espacio de hiperparámetros.

Para Louvain y Leiden, se han ejecutado, respectivamente, \(150\) iteraciones, explorando resoluciones entre \(0.1\) y \(2.0\) que maximicen tanto el coeficiente Q como el puntuaje FES, descritos en la Sección \ref{sec:metricas}. Los datos se han almacenado en una base de datos SQLite, lo que permite extraer informes tabulados y reanudar el ajuste desde el punto de guardado si se desea, facilitando la gestión y continuidad del proceso.


Como se comentó anteriormente, el principal objetivo del clustering en nuestra red PPI es determinar las comunidades funcionales subyacentes con la finalidad de identificar rutas metabólicas compatibles con la elaboración de dianas o biomarcadores para el tratamiento de la FTD. Por esto, no se debe perder la interpretabilidad biológica de la red, a la vez que se intenta maximizar la calidad del clustering aplicado. Dadas estas condiciones, se propone una optimización multiobjetivo, repartiendo las tareas de significancia biológica y calidad del clustering entre las siguientes métricas:

\begin{itemize}
	\item \textit{Modularidad (Q)}. Destinada a la evaluación de la calidad del clustering aplicado. Valores altos indican una conectividad interna densa dentro de los clústeres y conexiones dispersas entre ellos. La alta modularidad sugiere comunidades bien definidas.
	
	\item \textit{Puntaje de Enriquecimiento Funcional}: Para cada clúster, se realizó un análisis de enriquecimiento funcional mediante STRINGdb. Este puntaje refleja la significancia estadística de pathways, términos de la Ontología de Genes u otras anotaciones funcionales enriquecidas dentro de los clústeres. Un puntaje alto indica mayor relevancia biológica, ya que los clústeres enriquecidos en términos específicos probablemente representan procesos biológicos significativos.
\end{itemize}


\subsubsection*{Optimización}
Se generaron varias configuraciones de modelos al ajustar sistemáticamente los parámetros de los algoritmos y evaluamos cada configuración en base a modularidad y puntaje de enriquecimiento funcional. Al trazar estos puntajes en un frente de Pareto, identificamos las configuraciones que representan el mejor equilibrio entre coherencia estructural (modularidad) y relevancia biológica (enriquecimiento funcional) \cite{goodarzi2014PARETOFRONT1,jahan2013multiPARETOFRONT2,costa2015paretoPARETOFRONT3}. Las configuraciones a lo largo de esta frontera representan soluciones Pareto-eficientes de clustering, donde mejorar una métrica no compromete significativamente la otra, permitiendo explorar soluciones con un balance diferente de las métricas de rendimiento elegidas.

% \textbf{NOTA:} Podríamos aplicar algún algoritmo interesante de ajuste de los hiperparámetros, yo (Mario) tengo experiencia aplicando Tree-structured Parzen Estimator y otros algoritmos de ajuste Bayesiano en Python.

\begin{figure}[h]
	\centering
	\includegraphics[width=1\linewidth]{figures/methods/Flujo_de_trabajo.pdf}
	\caption{Flujo de trabajo que se seguirá durante el desarrollo del proyecto, partiendo del fenotipo como dato de entrada, metodología y análisis de resultados.}
	\label{fig:flujo_trabajo}
\end{figure}
