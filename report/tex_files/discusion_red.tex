En la red que se analizado y posteriormente utilizada y cómo se ha mencionado en la sección anterior hay 10 genes que no están presentas con respecto a los obtenidos en HPO debido al filtrado realizado.
Algunos de estos genes como DAO, PRKAR1B, ZDYVE26 parecen tener interacciones con confianzas muy bajas con el resto de genes de la red, hay que disminuir considerablemente este umbral para que aparezcan. 
Otros como GLT8D1, que tampoco aparecen con umbrales mayores que 500, no parece tener una demostrada asociación con enfermedades neurológicas \cite{Yilihamu2021}.
Sin embargo, otros genes con no tan altas puntuaciones de interacción como PLA2G6 y UNC13A parecen tener según la bibliografía una relación relevante con enfermedades neurodegenerativas que presentan la demencia frontotemporal. En el caso de PLA2G6  pese a no tener un rol principal en FTD pueden contribuir a su desarrollo temprano \cite{Tomiyama2011}. 
Por su parte UNC13A esta asociado al riesgo de padecer tanto ALS como FTP, y esta relacionado con el tejido neuronal \cite{Diekstra2012}.
Las bajas puntuaciones de interacción no necesariamente indican la falta de relevancia funcional con respecto al fenotipo. Pero si una falta de estudio y datos que demuestren su interacción con otros genes relevantes en el contexto del fenotipo.


En el análisis de la red hay dos genes destacaban con respecto a los demás, FUS y TARDBP, por su alta conectividad, centralidad y cercanía.
 Mutaciones patogénicas de TARDBP se presentan en enfermedades como la esclerosis lateral amiotrófica o la demencia frontotemporal, enfermedades en relacionadas con el fenotipo de estudio \cite{Monohar2009}.
 Este gen codifica para la proteína TDP-34, relacionada con las células nerviosas, la relación de este gen con demencias degenerativas.\cite{Ure2021}
 En estas enfermedades también parece común la identificación del mutaciones en el gen FUS.
La proteína que codifica FUS parece tener también un papel importante junto a TDP-43 en el transporte de RNAm, mantenimiento axonal y desarrollo de neuronas motoras  \cite{Lattante2013}.





