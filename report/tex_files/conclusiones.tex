\section{Conclusiones}

A lo largo de este trabajo, se ha hecho un estudio desde el punto de vista de la biología de sistemas sobre los genes relacionados con el fenotipo de demencia frontotemporal. Usando técnicas de clusterización y optimización multi-objetivo, se ha llegado a un mejor resultado, dado por el algoritmo de Leiden, que contiene las comunidades de genes sobre las cuales se realizó un análisis funcional. El resultado del análisis, relacionó los genes estudiados con algunas funciones biológicas, consistentes con estudios previos, como procesos relacionados con la $\beta$-amiloide, inflamación y activación microglial, etc. Esto demuestra que se pueden identificar procesos biológicos empleando la metodología usada. Además, se ha identificado un gen clave el PON1, cuya alteración está relacionada con un mayor riesgo de enfermedades neurodegenerativas. Esto valida el cumplimiento de los objetivos expuestos al inicio de este documento.
