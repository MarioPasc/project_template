\section{Introducción}

La demencia se define como un síndrome caracterizado por un deterioro cognitivo que produce alteraciones en la memoria, el pensamiento y el comportamiento de una persona. Esto dificulta la capacidad del paciente para realizar sus actividades sociales o laborales. \cite{Formiga2009} Se estima que hay alrededor de 44 millones de personas con demencia, se prevé que esta cifra será más del triple para 2050. \cite{Long2023}.
La \textit{Enfermedad de Alzheimer} (EA), es la enfermedad más común donde se presenta este síndrome (45-55\%), seguida de la demencia vascular y la demencia por cuerpos de Lewy. La demencia frontotemporal no supera el 5\% en las frecuencias relativas. \cite{ GOODMAN201728, GarreOlmo2016}. En este proyecto se estudiará un fenotipo concreto presente en varios casos de demencia, la denominada \textit{Demencia Frontotemporal} (FTD).
