\section{Introducción}

La demencia se define como un síndrome caracterizado por un deterioro cognitivo que produce alteraciones en la memoria, el pensamiento y el comportamiento de una persona. Esto dificulta la capacidad del paciente para realizar sus actividades sociales o laborales. \cite{Formiga2009} Se estima que hay alrededor de 44 millones de personas con demencia, se prevé que esta cifra será más del triple para 2050. \cite{Long2023}.
La \textit{Enfermedad de Alzheimer} (EA), es la enfermedad más común donde se presenta este síndrome (45-55\%), seguida de la demencia vascular y la demencia por cuerpos de Lewy. La demencia frontotemporal no supera el 5\% en las frecuencias relativas. \cite{ GOODMAN201728, GarreOlmo2016}. En este proyecto se estudiará un fenotipo concreto presente en varios casos de demencia, la denominada \textit{Demencia Frontotemporal} (FTD).


La FTD, al tratarse de un conjunto heterogéneo de fenotipos, muestra conexiones significativas con otras enfermedades neurodegenerativas. En estudios de coocurrencia de términos, \textit{Esclerosis Lateral Amiotrófica} (ELA) y EA son conceptos frecuentemente asociados con FTD, lo que sugiere una relación estrecha \cite{fneur.2024.1399600}. %sigue parte de Mario

El ejemplo más sonado en la literatura se relaciona con la ELA, una forma común de enfermedad de la motoneurona (MND) \cite{NHS_MND}, la cual es una enfermedad neurodegenerativa que comparte causas genéticas y neuropatologías con la FTD \cite{10.1093/brain/awr195}. Mutaciones en genes como \textit{C9orf72} \cite{DeJesusHernandez2011}, \textit{TARDBP} \cite{Arai2006} y \textit{OPTN} \cite{Bussi2018} se han identificado en pacientes que presentan el fenotipo FTD, padecen ELA, o con manifestaciones de ambas.  Las expansiones en \textit{C9orf72} una causa frecuente en ambos casos \cite{Balendra2018, DeJesusHernandez2011}. Patológicamente, se han observado disfunciones en el sistema autofagia-lisosoma \cite{Casterton2020} e inclusiones citoplásmicas neuronales tau-negativas pero ubiquitina-positivas \cite{Arai2006} en ambas enfermedades. Pacientes con MND pueden desarrollar afectación cognitiva y evolucionar a FTD \cite{Devenney2015}, e incluso mostrar síntomas típicos de bvFTD \cite{Devenney2019}. Asimismo, el parkinsonismo, especialmente el síndrome corticobasal, presenta solapamientos considerables con la FTD, incluyendo trastornos motores y cognitivos \cite{Orphanet_MND}

