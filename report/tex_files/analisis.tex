A partir de los resultados obtenidos y representados mediante gráficos de enriquecimiento (bar plot, dot plot y cnet plot), es posible identificar procesos clave relacionados con la neurodegeneración, la inflamación y el metabolismo neuronal. En esta sección, discutiremos los hallazgos más relevantes y su relación con estudios previos, evaluando sus implicaciones para la comprensión de la patogénesis de la DFT.

\subsection{Procesos Relacionados con el β-Amiloide}

Uno de los hallazgos más destacados en nuestro análisis es la implicación de procesos relacionados con el β-amiloide, incluyendo: Formación del β-amiloide (GO:0034205) y metabolismo del β-amiloide (GO:0050435). Estos procesos se observan claramente en la figura \ref{bar_plot} y en la figura \ref{dot_plot}, donde destacan por su alta significancia estadística y proporción de genes implicados.


\subsection{Inflamación y Activación Microglial}

La presencia de procesos relacionados con la activación de células microgliales (GO:0001774) y la regulación positiva de la fagocitosis (GO:0060100) se ilustra en la figura \ref{bar_plot} y en la figura \ref{dot_plot}. Estos resultados reflejan una respuesta inflamatoria exacerbada que es característica de la neurodegeneración en la DFT.


\subsection{Procesamiento del Receptor Notch}

El procesamiento de receptores Notch (GO:0035333) está representado en la figura \ref{bar_plot} y en la figura \ref{dot_plot}. La vía de señalización Notch es esencial para la diferenciación neuronal y su disfunción podría comprometer la homeostasis neuronal.

\subsection{Alteraciones Metabólicas y de Transporte Neuronal}

Los procesos de transporte axo-dendrítico (GO:0008088) y transporte de ácidos nucleicos (GO:0050657) se identifican en la figura \ref{dot_plot}. Estas alteraciones también se representan en la figura \ref{cnet_plot}, donde se observa la conexión de genes específicos con estos procesos.

\subsection{Rol de la Paraoxonasa 1 (PON1)}

Los procesos metabólicos y de transporte neuronal son clave en la patogénesis de la DFT. En este contexto, PON1 (Paraoxonasa 1) destaca en la figura \ref{cnet_plot} por participar en dos rutas metabólicas distintas: el catabolismo de compuestos aromáticos (GO:0019439) y el catabolismo de ácidos orgánicos (GO:0016054). Esta participación sugiere que PON1 actúa como un punto de convergencia funcional en el metabolismo antioxidante y de lípidos.