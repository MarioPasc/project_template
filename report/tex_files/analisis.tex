A partir de los resultados obtenidos y representados mediante gráficos de enriquecimiento (Figura~\ref{fig:bar_plot}, Figura~\ref{fig:dot_plot} y Figura~\ref{fig:cnet_plot}), es posible identificar procesos clave relacionados con la neurodegeneración, la inflamación y el metabolismo neuronal. En esta sección, discutiremos los hallazgos más relevantes y su relación con estudios previos, evaluando sus implicaciones para la comprensión de la patogénesis de la DFT.

\subsection{Procesos Relacionados con el \(\beta\)-Amiloide}

Uno de los hallazgos más destacados en nuestro análisis es la implicación de procesos relacionados con el \(\beta\)-amiloide, incluyendo formación del \(\beta\)-amiloide (GO:0034205) y metabolismo del \(\beta\)-amiloide (GO:0050435). Estos procesos se observan claramente en la Figura~\ref{fig:bar_plot} y la Figura~\ref{fig:dot_plot}, donde destacan por su alta significancia estadística y proporción de genes implicados.

Estos procesos son críticos porque la acumulación de \(\beta\)-amiloide se asocia típicamente con la enfermedad de Alzheimer, pero también se ha descrito en la DFT en subtipos específicos y en casos mixtos \cite{hardy2002amyloid}. La alteración en el metabolismo del \(\beta\)-amiloide podría contribuir al daño neuronal y a la disfunción sináptica, exacerbando los síntomas clínicos de la DFT.

Además, la regulación del aclaramiento del \(\beta\)-amiloide (GO:1900221) y la regulación positiva del aclaramiento del \(\beta\)-amiloide (GO:1900223) están claramente representadas en la Figura~\ref{fig:cnet_plot}, lo que sugiere que los mecanismos encargados de eliminar estas proteínas tóxicas pueden estar fallando \cite{heneka2015neuroinflammation}.


\subsection{Inflamación y Activación Microglial}

La presencia de procesos relacionados con la activación de células microgliales (GO:0001774) y la regulación positiva de la fagocitosis (GO:0060100) se ilustra en la Figura~\ref{fig:bar_plot} y la Figura~\ref{fig:dot_plot}. Estos resultados reflejan una respuesta inflamatoria exacerbada característica de la neurodegeneración en la DFT.

La conexión de genes como TREM2 con estos procesos inflamatorios puede observarse en la Figura~\ref{fig:cnet_plot}. La activación crónica de la microglía puede resultar en la liberación de citocinas proinflamatorias, lo que agrava la neurodegeneración \cite{yeh2016trem2}.

\subsection{Procesamiento del Receptor Notch}

El procesamiento de receptores Notch (GO:0035333) está representado en la Figura~\ref{fig:bar_plot} y la Figura~\ref{fig:dot_plot}. La vía de señalización Notch es esencial para la diferenciación neuronal y su disfunción podría comprometer la homeostasis neuronal \cite{kopan2009canonical}.

En la Figura~\ref{fig:cnet_plot}, se puede observar que genes como PSEN1 y PSEN2 están directamente conectados con estos procesos, lo que refuerza la hipótesis de que alteraciones en esta vía contribuyen a la degeneración neuronal en la DFT.

\subsection{Alteraciones Metabólicas y de Transporte Neuronal}

Los procesos de transporte axo-dendrítico (GO:0008088) y transporte de ácidos nucleicos (GO:0050657) se identifican en la Figura~\ref{fig:dot_plot}. Estas alteraciones también se representan en la Figura~\ref{fig:cnet_plot}, donde se observa la conexión de genes específicos con estos procesos.

Asimismo, los procesos de catabolismo de compuestos aromáticos (GO:0019439) y catabolismo de ácidos orgánicos (GO:0016054) están claramente reflejados en la Figura~\ref{fig:bar_plot}, sugiriendo disfunciones metabólicas que podrían desempeñar un papel importante en la neurodegeneración \cite{seelaar2011clinical}.

\subsection{Rol de la Paraoxonasa 1 (PON1)}

Los procesos metabólicos y de transporte neuronal son clave en la patogénesis de la DFT. En este contexto, PON1 (Paraoxonasa 1) destaca en la Figura~\ref{fig:cnet_plot} por participar en dos rutas metabólicas distintas: el catabolismo de compuestos aromáticos (GO:0019439) y el catabolismo de ácidos orgánicos (GO:0016054). Esta participación sugiere que PON1 actúa como un punto de convergencia funcional en el metabolismo antioxidante y de lípidos.

Además, PON1 es un nodo crítico en la red de interacción proteína-proteína (PPI), lo que refuerza su importancia en el mantenimiento de la homeostasis neuronal. Su disfunción podría agravar el estrés oxidativo y el daño celular, procesos asociados a la neurodegeneración en la DFT \cite{costa2005paraoxonase}.

Estos hallazgos sugieren que modular la actividad de PON1 podría ser una estrategia terapéutica para mitigar el daño oxidativo y mejorar el metabolismo neuronal en pacientes con DFT.