\documentclass{bmcart}

%%%%%%%%%%%%%%%%%%%%%%%%%%%%%%%%%%%%%%%%%%%%%%
%%                                          %%
%% CARGA DE PAQUETES DE LATEX               %%
%%                                          %%
%%%%%%%%%%%%%%%%%%%%%%%%%%%%%%%%%%%%%%%%%%%%%%

%%% Load packages
\usepackage{amsthm,amsmath}
\usepackage{graphicx}
%\RequirePackage[numbers]{natbib}
%\RequirePackage{hyperref}
\usepackage[utf8]{inputenc} %unicode support
%\usepackage[applemac]{inputenc} %applemac support if unicode package fails
%\usepackage[latin1]{inputenc} %UNIX support if unicode package fails
\usepackage{array}
\usepackage{booktabs}  % Paquete instalado
\usepackage{caption}   % Paquete instalado
\usepackage{multirow}


%%%%%%%%%%%%%%%%%%%%%%%%%%%%%%%%%%%%%%%%%%%%%%
%%                                          %%
%% COMIENZO DEL DOCUMENTO                   %%
%%                                          %%
%%%%%%%%%%%%%%%%%%%%%%%%%%%%%%%%%%%%%%%%%%%%%%

\begin{document}

	\begin{frontmatter}
	
		\begin{fmbox}
			\dochead{Research}
			
			%%%%%%%%%%%%%%%%%%%%%%%%%%%%%%%%%%%%%%%%%%%%%%
			%% INTRODUCIR TITULO PROYECTO               %%
			%%%%%%%%%%%%%%%%%%%%%%%%%%%%%%%%%%%%%%%%%%%%%%
			
			\title{Explorando la Demencia Frontotemporal mediante Biología de Sistemas: Un Enfoque Integrado}
			
			%%%%%%%%%%%%%%%%%%%%%%%%%%%%%%%%%%%%%%%%%%%%%%
			%% AUTORES. METER UNA ENTRADA AUTHOR        %%
			%% POR PERSONA                              %%
			%%%%%%%%%%%%%%%%%%%%%%%%%%%%%%%%%%%%%%%%%%%%%%
			
		%	\author[
		%	  addressref={aff1},                   % ESTA LINEA SE COPIA IGUAL PARA CADA AUTOR
		%	  corref={aff1},                       % ESTA LINEA SOLO DEBE TENERLA EL COORDINADOR DEL GRUPO
		%	  email={jane.e.doe@cambridge.co.uk}   % VUESTRO CORREO ACTIVO
		%	]{\inits{J.E.}\fnm{Jane E.} \snm{Doe}} % inits: INICIALES DE AUTOR, fnm: NOMBRE DE AUTOR, snm: APELLIDOS DE AUTOR
		% ALUMNO COORDINADOR PRIMERO
			\author[
			addressref={aff1},
			email={pascualgonzalez.mario@uma.es}
			]{\inits{M.P.G}\fnm{Mario} \snm{Pascual González}}
			\author[
			  addressref={aff1},
			  email={ainhoasb@uma.es}
			]{\inits{A.N.S.B.}\fnm{Ainhoa Nerea} \snm{Santana Bastante}}
			\author[
				addressref={aff1},
				email={0610901647@uma.es},
			]{\inits{C.R.G.}\fnm{Carmen}\snm{ Rodríguez González}}
			\author[
			addressref={aff1},
			email={gonzalo.mesas.aranda@uma.es}
			]{\inits{G.M.A}\fnm{Gonzalo} \snm{Mesas Aranda}}
			\author[
			addressref={aff1},
			email={ainhoapergon@uma.es}
			]{\inits{A.P.G.}\fnm{Ainhoa} \snm{Pérez González}}
			
			%%%%%%%%%%%%%%%%%%%%%%%%%%%%%%%%%%%%%%%%%%%%%%
			%% AFILIACION. NO TOCAR                     %%
			%%%%%%%%%%%%%%%%%%%%%%%%%%%%%%%%%%%%%%%%%%%%%%
			
			\address[id=aff1]{%                           % unique id
			  \orgdiv{ETSI Informática},             % department, if any
			  \orgname{Universidad de Málaga},          % university, etc
			  \city{Málaga},                              % city
			  \cny{España}                                    % country
			}
		
		\end{fmbox}% comment this for two column layout
		
		\begin{abstractbox}
		
			\begin{abstract} % abstract
			
			%%%%%%%%%%%%%%%%%%%%%%%%%%%%%%%%%%%%%%%%%%%%%%%
			%% RESUMEN BREVE DE NO MAS DE 100 PALABRAS   %%
			%%%%%%%%%%%%%%%%%%%%%%%%%%%%%%%%%%%%%%%%%%%%%%%	
			En este trabajo se ha construido una red de interacción proteína-proteína a partir de los genes asociados a dicho fenotipo en la ontología HPO. Sobre esta red se han aplicado tres algoritmos de agrupamiento (Leiden, Louvain y Fast Greedy) para identificar módulos funcionales.

			La evaluación del rendimiento de estos algoritmos y la optimización de sus parámetros se ha realizado empleando la modularidad, junto con una métrica que permite cuantificar la relevancia biológica de los clústeres generados. Finalmente, a partir de los módulos resultantes, se llevó a cabo un análisis de enriquecimiento funcional con el objetivo de identificar procesos biológicos clave relacionados con la demencia frontotemporal.
			\end{abstract}
			
			%%%%%%%%%%%%%%%%%%%%%%%%%%%%%%%%%%%%%%%%%%%%%%
			%% PALABRAS CLAVE DEL PROYECTO              %%
			%%%%%%%%%%%%%%%%%%%%%%%%%%%%%%%%%%%%%%%%%%%%%%
			
			\begin{keyword}
			\kwd{HPO}
			\kwd{Demencia Frontotemporal}
			\kwd{Biología de Sistemas}
			\end{keyword}
		
		
		\end{abstractbox}
	
	\end{frontmatter}
	
	%%%%%%%%%%%%%%%%%%%%%%%%%%%%%%%%%
	%% COMIENZO DEL DOCUMENTO REAL %%
	%%%%%%%%%%%%%%%%%%%%%%%%%%%%%%%%%
	
	\section{Introducción}

La demencia se define como un síndrome caracterizado por un deterioro cognitivo que produce alteraciones en la memoria, el pensamiento y el comportamiento de una persona. Esto dificulta la capacidad del paciente para realizar sus actividades sociales o laborales. \cite{Formiga2009} Se estima que hay alrededor de 44 millones de personas con demencia, se prevé que esta cifra será más del triple para 2050. \cite{Long2023}.
La \textit{Enfermedad de Alzheimer} (EA), es la enfermedad más común donde se presenta este síndrome (45-55\%), seguida de la demencia vascular y la demencia por cuerpos de Lewy. La demencia frontotemporal no supera el 5\% en las frecuencias relativas. \cite{ GOODMAN201728, GarreOlmo2016}. En este proyecto se estudiará un fenotipo concreto presente en varios casos de demencia, la denominada \textit{Demencia Frontotemporal} (FTD).


La FTD, al tratarse de un conjunto heterogéneo de fenotipos, muestra conexiones significativas con otras enfermedades neurodegenerativas. En estudios de coocurrencia de términos, \textit{Esclerosis Lateral Amiotrófica} (ELA) y EA son conceptos frecuentemente asociados con FTD, lo que sugiere una relación estrecha \cite{fneur.2024.1399600}. %sigue parte de Mario

El ejemplo más sonado en la literatura se relaciona con la ELA, una forma común de enfermedad de la motoneurona (MND) \cite{NHS_MND}, la cual es una enfermedad neurodegenerativa que comparte causas genéticas y neuropatologías con la FTD \cite{10.1093/brain/awr195}. Mutaciones en genes como \textit{C9orf72} \cite{DeJesusHernandez2011}, \textit{TARDBP} \cite{Arai2006} y \textit{OPTN} \cite{Bussi2018} se han identificado en pacientes que presentan el fenotipo FTD, padecen ELA, o con manifestaciones de ambas.  Las expansiones en \textit{C9orf72} una causa frecuente en ambos casos \cite{Balendra2018, DeJesusHernandez2011}. Patológicamente, se han observado disfunciones en el sistema autofagia-lisosoma \cite{Casterton2020} e inclusiones citoplásmicas neuronales tau-negativas pero ubiquitina-positivas \cite{Arai2006} en ambas enfermedades. Pacientes con MND pueden desarrollar afectación cognitiva y evolucionar a FTD \cite{Devenney2015}, e incluso mostrar síntomas típicos de bvFTD \cite{Devenney2019}. Asimismo, el parkinsonismo, especialmente el síndrome corticobasal, presenta solapamientos considerables con la FTD, incluyendo trastornos motores y cognitivos \cite{Orphanet_MND}


	\section{Materiales y métodos}

\subsection{Datos}

\subsection{Software}

Para el análisis funcional y la construcción de redes genéticas en este estudio, se seleccionaron herramientas especializadas que permiten tanto la exploración bioinformática como la visualización de datos complejos. Dado que el objetivo principal es investigar la interacción entre genes y módulos específicos asociados a la demencia frontotemporal, se ha optado por una combinación de paquetes en Python y R que ofrecen un balance entre precisión analítica y capacidades visuales avanzadas.

\subsubsection*{Paquetes de Python para el análisis funcional y otras funciones}

\begin{itemize}
	\item \textbf{Pandas (versión 2.2.3)}: Estructura de datos flexible para manipulación de datos; ideal para manejar resultados de enriquecimiento \cite{pandas}.
	\item \textbf{Matplotlib (versión 3.8.1)}: Librería de visualización de datos, utilizada para crear gráficos en 2D como líneas, barras y dispersión \cite{matplotlib}.
	\item \textbf{Scienceplots (versión 2.1.1)}: Paquete que proporciona estilos de gráficos preconfigurados para facilitar la creación de visualizaciones con estética científica y de alta calidad \cite{scienceplots}.
	\item \textbf{Requests (versión 2.31.0)}: Biblioteca para realizar solicitudes HTTP, utilizada para conectar con la API de STRINGdb \cite{requests}.
	\item \textbf{GOATOOLS (versión 1.2.3)}: Paquete para realizar análisis de enriquecimiento funcional de términos Gene Ontology (GO) \cite{goatools}.
	\item \textbf{G:Profiler (versión 1.4.0)}: Herramienta para análisis de enriquecimiento que abarca GO, KEGG y Reactome, entre otras bases de datos \cite{gprofiler}.
	\item \textbf{Statsmodels (versión 0.14.0)}: Biblioteca de estadísticas para aplicar ajustes de p-valor (como Benjamini-Hochberg) en los resultados \cite{statsmodels}.
\end{itemize}

\vspace{1cm}

\subsubsection*{Paquetes de R para clustering y visualización de redes}

\begin{itemize}
	\item \textbf{Cluster (versión 2.1.5)}: Proporciona algoritmos clásicos de clustering como k-means y clustering jerárquico \cite{cluster}.
	\item \textbf{Factoextra (versión 1.0.8)}: Extensión para la visualización de resultados de clustering de manera intuitiva y clara \cite{factoextra}.
	\item \textbf{WGCNA (versión 1.71)}: Utilizado para identificar módulos de genes coexpresados, especialmente en estudios de expresión génica \cite{wgcna}.
	\item \textbf{Igraph (versión 1.5.0)}: Paquete para construir, analizar y visualizar redes genéticas y módulos de coexpresión \cite{igraph}.
	\item \textbf{STRINGdb (versión 2.10.0)}: Conector para la base de datos STRING, permite realizar enriquecimiento funcional y visualizar redes \cite{stringdb}.
	\item \textbf{Pathview (versión 1.36.0)}: Herramienta para representar gráficamente rutas de KEGG con datos de expresión génica \cite{pathview}.
	\item \textbf{ClusterProfiler (versión 4.15.0)}: Paquete para análisis de enriquecimiento funcional en términos GO, KEGG y Reactome, con opciones de ajuste \cite{clusterprofiler}.
\end{itemize}

\subsection{Análisis Funcional}

\subsection{Clustering}

Al aplicar algoritmos de clustering, nuestro objetivo es descubrir comunidades funcionales dentro de la red de genes asociada a la demencia frontotemporal. Estas comunidades, módulos, o \textit{clusters} funcionales pueden representar procesos biológicos específicos, vías celulares, o mecanismos asociados al fenotipo FTD. Encontrar estos clusters podría revelar posibles dianas terapéuticas o grupos de biomarcadores dentro del conjunto de genes, lo cual podría abrir las puertas a nuevos tratamientos para los pacientes de FTD.

\subsubsection*{Red}

En la red de interacción proteína-proteína (PPI) obtenida mediante la API de STRINGdb, cada nodo representa un gen y cada arista representa una interacción entre genes, ponderada por un puntaje de confianza otorgado por STRINGdb \cite{szklarczyk2023stringdb}. Para enfocarnos en interacciones más sólidas y confiables, se aplicó un umbral mínimo de confianza en las aristas, lo que permitió refinar la red y mejorar la relevancia de los clústeres detectados 

% (\textbf{NOTA:} Se debe ahondar en esto si al final se hace: ¿Por qué ese umbral? ¿Y si fueran otros umbrales?).

\subsubsection*{Algoritmos}

A continuación, se detallan los tres algoritmos de clustering, proporcionados por la librería \textit{iGraph}, elegidos para este estudio, los cuales pretenden cubrir diferentes enfoques teóricos en la detección de comunidades funcionales \cite{igraph}.  

\textit{Walktrap}: Algoritmo de clustering jerárquico aglomerativo que detecta comunidades basándose en la probabilidad de que recorridos aleatorios dentro de la red permanezcan dentro de clústeres cohesivos \cite{pons2005walktrap}. El algoritmo aplica recorridos aleatorios para capturar la proximidad de los nodos, asumiendo que los nodos que pertenecen a la misma comunidad tienen mayores probabilidades de estar conectados mediante recorridos cortos. Comienza con cada nodo como su propio clúster, fusionando sucesivamente los clústeres cercanos hasta alcanzar una partición óptima. Se ajustó la longitud de los pasos (\textit{steps}).

\textit{Algoritmo de Leiden}: Se basa en optimizar el Constant Potts Model (CPM) de la red \cite{traag2019leiden,constantplottsmodel}. Esta función, mostrada en la Ecuación \ref{eq:cpm}, evalúa la calidad de una partición en comunidades considerando un balance entre la densidad interna de cada clúster y el parámetro de resolución \(\gamma\), que controla el tamaño preferido de las comunidades finales. El término \(n_c\) se refiere al número de nodos, mientras que \(e_c\) es el número de aristas internas, ambos referidos a la comunidad \(c\).

\begin{equation}
	\label{eq:cpm}
	\mathcal{H} = \sum_{c} \left[ e_c - \gamma \left( \frac{n_c (n_c - 1)}{2} \right) \right]
\end{equation}

\noindent Se ajustó el parámetro \(\gamma\), así como el número de iteraciones del algoritmo, permitiendo que el Leiden refinara iterativamente la partición. El resto de parámetros se fijaron a su valor por defecto. 

% (\textbf{NOTA:} Hay un parámetro intersante, \textit{initial\_membership} el cual son nodos 'semilla' que se pasan como argumento, y el algoritmo intenta mejorar las comunidades alrededor de estos nodos. Podríamos usar genes del análisis funcional como semilla y ver qué pasa.)

\textit{Algoritmo de Louvain}: Este algoritmo optimiza la modularidad de la red, una métrica que mide la calidad de una partición en comunidades al comparar la densidad de conexiones internas con la densidad esperada si las conexiones fueran aleatorias \cite{Blondel2008Louvain}. La modularidad \( Q \), definida en la Ecuación \ref{eq:modularity}, es un valor escalar entre \(-1\) y \(1\) que representa la diferencia entre la densidad de aristas dentro de las comunidades y la densidad de aristas esperada. El algoritmo Louvain maximiza esta modularidad en dos fases, cuyo resultado es la formación una nueva red sobre la cual se repite el proceso hasta que la modularidad no mejore más. En redes ponderadas, los autores del algoritmo expresaron la modularidad a optimizar como:

\begin{equation}
	\label{eq:modularity}
	Q = \frac{1}{2m} \sum_{i,j} \left[ A_{ij} - \frac{k_i k_j}{2m} \right] \delta(c_i, c_j),
\end{equation}

\noindent donde \( A_{ij} \) es el peso de la arista entre los nodos \( i \) y \( j \), \( k_i \) y \( k_j \) son las sumas de los pesos de las aristas conectadas a los nodos \( i \) y \( j \), \( c_i \) y \( c_j \) representan las comunidades a las que pertenecen los nodos \( i \) y \( j \), y \( \delta(c_i, c_j) \) es una función delta de Kronecker que es 1 si \( c_i = c_j \) y 0 en caso contrario. El término \( m \) es la suma total de los pesos de las aristas en la red.

\noindent Se ajustó el parámetro de resolución, que controla el tamaño final de las comunidades. El resto de parámetros se dejaron con sus valores por defecto.

\subsubsection*{Medidas de Rendimiento}

\subsubsection*{Optimización}
	\newpage
	
\section{Resultados}
Se ha explorado distintas técnicas de biología de sistemas con el objetivo de entender el fenotipo \textit{Frontotemporal Dementia} y sus relaciones con genes y las funciones de estas. 
Mediante el análisis de redes de interacción proteína-proteína, de los genes relacionados con el fenotipo, obtenidos mediante StringDB.

\subsection{Red PPI y sus propiedades}

\input{tex_files/ned.tex}


	\newpage
	\section{Discusión}

En este estudio se ha llevado a cabo un análisis de clustering con dos configuraciones del algoritmo de Leiden: una optimizada para máxima modularidad y otra para máximo puntaje de enriquecimiento funcional. Para demostrar que ambas configuraciones producen resultados muy similares, se han comparado los resultados utilizando un diagrama de Venn (Figura~\ref{fig:venn_plot}). Este diagrama destaca los procesos biológicos comunes entre ambas configuraciones, así como los exclusivos de cada enfoque.

Se observa la existencia de 22 términos compartidos entre los 24 identificados para cada configuración del algoritmo de Leiden, lo que refuerza los hallazgos del análisis funcional. Los procesos comunes representan rutas biológicas esenciales que son robustas frente a las variaciones metodológicas, sugiriendo que reflejan aspectos centrales del fenotipo estudiado. Esto no solo valida la calidad de los resultados obtenidos, sino que también  subraya la importancia de integrar diferentes enfoques para obtener una visión más completa y confiable del sistema biológico en análisis.

En esta sección, discutiremos los resultados obtenidos del análisis funcional que se realizó únicamente con la configuración de máximo puntaje de enriquecimiento funcional. Dado que, como se ha señalado previamente, el diagrama de Venn (Figura~\ref{fig:venn_plot}) refuerza la confianza de los siguientes hallazgos, ya que muestra una gran similitud con los resultados obtenidos bajo la configuración de máxima modularidad. Por esta razón, la elección de una única configuración asegura una interpretación robusta y confiable.

A partir de los resultados obtenidos y representados mediante gráficos de enriquecimiento (Figura~\ref{fig:bar_plot}, Figura~\ref{fig:dot_plot} y Figura~\ref{fig:cnet_plot}), es posible identificar procesos clave relacionados con la neurodegeneración, la inflamación y el metabolismo neuronal. A continuación, se discutirán los hallazgos más relevantes y su relación con estudios previos, evaluando sus implicaciones para la comprensión de la patogénesis de la DFT.

\subsection{Procesos Relacionados con el \(\beta\)-Amiloide}

Uno de los hallazgos más destacados del análisis es la implicación de procesos relacionados con el \(\beta\)-amiloide, como la formación (GO:0034205) y el metabolismo del \(\beta\)-amiloide (GO:0050435). Estos procesos, evidentes en las Figuras~\ref{fig:bar_plot} y~\ref{fig:dot_plot}, son relevantes por su alta significancia estadística y el número de genes implicados.

Aunque la acumulación de \(\beta\)-amiloide es característica de la enfermedad de Alzheimer, estudios recientes han identificado su presencia en subtipos específicos de DFT y en casos mixtos \cite{hardy2002amyloid, ling2010frontotemporal}. La disfunción en el metabolismo del \(\beta\)-amiloide podría contribuir al daño neuronal y a la disfunción sináptica, agravando los síntomas de la DFT \cite{selkoe2002alzheimers}.

Además, procesos de aclaramiento del \(\beta\)-amiloide (GO:1900221) observados en la Figura~\ref{fig:cnet_plot} sugieren fallos en la eliminación de estos péptidos tóxicos \cite{heneka2015neuroinflammation}. La alteración en el aclaramiento puede estar asociada a una activación microglial crónica y a una respuesta inflamatoria exacerbada, que potencian la neurodegeneración \cite{chen2016microglia}.

Estos resultados resaltan la posible convergencia patogénica entre la EA y la DFT, sugiriendo que modular el metabolismo y aclaramiento del \(\beta\)-amiloide podría ser una estrategia terapéutica a considerar para ciertos subtipos de DFT.


\subsection{Inflamación y Activación Microglial}

El análisis revela una implicación significativa de procesos relacionados con la activación de células microgliales (GO:0001774) y la regulación positiva de la fagocitosis (GO:0060100), como se muestra en las Figuras~\ref{fig:bar_plot} y~\ref{fig:dot_plot}. La activación microglial crónica es una característica común en la DFT y otras taupatías, contribuyendo a la progresión de la neurodegeneración \cite{heneka2015neuroinflammation, rajendran2009microglia}.

En este estudio, genes como TREM2 destacan en la Figura~\ref{fig:cnet_plot} por su papel central en la activación microglial y la respuesta inmunitaria innata. Mutaciones en TREM2 están asociadas con un mayor riesgo de DFT y otras enfermedades neurodegenerativas, facilitando una respuesta inflamatoria desregulada \cite{yeh2016trem2, ulland2017trem2}.

La activación prolongada de la microglía puede conducir a la liberación de citocinas proinflamatorias como IL-1\(\beta\), IL-6 y TNF-\(alpha\), lo que exacerba el daño neuronal y sináptico \cite{block2007microglia}. Además, la fagocitosis microglial desregulada puede interferir con el aclaramiento de proteínas tóxicas, contribuyendo a la acumulación de agregados proteicos y promoviendo la neuroinflamación crónica \cite{lull2010microglial}.

Estos hallazgos sugieren que modular la actividad microglial o reducir la inflamación crónica podría representar una estrategia terapéutica prometedora para mitigar el daño neuronal en la DFT.


\subsection{Procesamiento del Receptor Notch}

El análisis también identifica procesos relacionados con el procesamiento del receptor Notch (GO:0035333), como se observa en las Figuras~\ref{fig:bar_plot} y~\ref{fig:dot_plot}. La vía de señalización Notch desempeña un papel fundamental en la diferenciación neuronal, la proliferación celular y el mantenimiento de la homeostasis cerebral \cite{kopan2009canonical}. La disfunción en esta vía puede comprometer la estabilidad estructural y funcional de las neuronas, contribuyendo a la neurodegeneración observada en la DFT \cite{abbott2019notch}.

En la Figura~\ref{fig:cnet_plot}, genes como PSEN1 y PSEN2 están claramente implicados en estos procesos. Estas proteínas forman parte del complejo de la  \(\gamma\)-secretasa, responsable del corte proteolítico del receptor Notch \cite{de2002presenilins}. Mutaciones en PSEN1 y PSEN2 no solo afectan el metabolismo del \(\beta\)-amiloide, sino que también interfieren en el procesamiento del receptor Notch, lo que puede llevar a una alteración en la señalización celular y a una disfunción neuronal progresiva \cite{sherrington1995cloning}.

Además, una señalización Notch alterada puede impactar negativamente en la neurogénesis y en la capacidad de reparación neuronal, procesos críticos para contrarrestar la neurodegeneración \cite{ables2011notch}. La convergencia de estos mecanismos patológicos sugiere que la disfunción del procesamiento Notch es una característica importante en la patogénesis de la DFT y podría ser un objetivo terapéutico potencial.


\subsection{Alteraciones Metabólicas y de Transporte Neuronal}

En el análisis se observaron procesos relacionados con el transporte axo-dendrítico (GO:0008088) y el transporte de ácidos nucleicos (GO:0050657), representados en las Figuras~\ref{fig:dot_plot} y~\ref{fig:cnet_plot}. Estos procesos son esenciales para el correcto funcionamiento de las neuronas, facilitando la distribución de proteínas, ARNm y organelos a lo largo de los axones y dendritas \cite{goldstein2001axonal}.

Las alteraciones en estos mecanismos pueden provocar una acumulación de proteínas mal plegadas y una disfunción sináptica, contribuyendo a la neurodegeneración observada en la DFT \cite{moreno2016axonal}. En particular, genes asociados con el citosqueleto y motores moleculares (como KIF5A y DYNC1H1) están implicados en estas vías y se destacan en la Figura~\ref{fig:cnet_plot} \cite{reid2019kif5a}.

Además, se observan alteraciones en procesos metabólicos como el catabolismo de compuestos aromáticos (GO:0019439) y el catabolismo de ácidos orgánicos (GO:0016054) en la Figura~\ref{fig:bar_plot}. Estos procesos son críticos para el mantenimiento del equilibrio energético y la eliminación de metabolitos tóxicos \cite{mccabe2001metabolism}. La disfunción en el metabolismo neuronal puede generar estrés oxidativo y acumular productos tóxicos, exacerbando la degeneración neuronal en la DFT \cite{seelaar2011clinical}.

Estos hallazgos sugieren que las estrategias terapéuticas dirigidas a mejorar el transporte neuronal y modular el metabolismo celular podrían ser beneficiosas para frenar la progresión de la DFT.


\subsection{Rol de la Paraoxonasa 1 (PON1)}

El análisis identifica a PON1 (Paraoxonasa 1) como un gen clave en los procesos metabólicos y de transporte neuronal, específicamente en el catabolismo de compuestos aromáticos (GO:0019439) y el catabolismo de ácidos orgánicos (GO:0016054), como se observa en la Figura~\ref{fig:cnet_plot}. La proteína PON1 participa en la detoxificación de compuestos oxidativos y en el metabolismo de lípidos, contribuyendo a proteger las neuronas del estrés oxidativo \cite{costa2005paraoxonase}.

La disfunción de PON1 puede conducir a una acumulación de productos tóxicos y a un aumento del daño oxidativo, procesos implicados en la neurodegeneración característica de la DFT \cite{kim2006paraoxonase}. Además, estudios sugieren que niveles bajos de actividad de PON1 están asociados con un mayor riesgo de enfermedades neurodegenerativas, debido a su incapacidad para neutralizar los radicales libres \cite{brophy2001paraoxonase}.

La red de interacción proteína-proteína (PPI) muestra que PON1 es un nodo central, lo que sugiere su participación en múltiples rutas metabólicas. Este papel destacado implica que modular la actividad de PON1 podría ser una estrategia terapéutica para reducir el daño oxidativo y mejorar la función metabólica en pacientes con DFT \cite{james2009paraoxonase}.


	\section{Conclusiones}

A lo largo de este trabajo, se ha hecho un estudio desde el punto de vista de la biología de sistemas sobre los genes relacionados con el fenotipo de demencia frontotemporal. Usando técnicas de clusterización y optimización multi-objetivo, se ha llegado a un mejor resultado, dado por el algoritmo de Leiden, que contiene las comunidades de genes sobre las cuales se realizó un análisis funcional. El resultado del análisis, relacionó los genes estudiados con algunas funciones biológicas, consistentes con estudios previos, como procesos relacionados con la $\beta$-amiloide, inflamación y activación microglial, etc. Esto demuestra que se pueden identificar procesos biológicos empleando la metodología usada. Además, se ha identificado un gen clave el PON1, cuya alteración está relacionada con un mayor riesgo de enfermedades neurodegenerativas. Esto valida el cumplimiento de los objetivos expuestos al inicio de este documento.

	\section{Anexo A: KDE mediante ventana deslizante de Parzen-Rosenblatt en el TPE}

\label{sec:anexo_a}

La Estimación de Densidad por Kernel (KDE) mediante el método de ventana deslizante de Parzen-Rosenblatt es una técnica no paramétrica para estimar la función de densidad de probabilidad (PDF) de una variable aleatoria a partir de una muestra de datos. La estimación en un punto \( x \) se realiza sumando las contribuciones de cada dato \( x_i \) mediante una función kernel \( K \) centrada en \( x_i \):

\begin{equation}
	\hat{f}(x) = \frac{1}{n h} \sum_{i=1}^{n} K\left( \frac{x - x_i}{h} \right)
\end{equation}


Donde \( n \) es el número de muestras, \( h \) es el ancho de banda (parámetro de suavizado) y \( K(u) \) es la función kernel, comúnmente el kernel gaussiano:

\begin{equation}
	K(u) = \frac{1}{\sqrt{2\pi}} e^{- \frac{u^2}{2}}
\end{equation}


En el contexto del TPE, este método se utiliza para estimar las distribuciones de probabilidad de los hiperparámetros en los conjuntos de buen rendimiento \( \mathcal{C}_1 \) y peor rendimiento \( \mathcal{C}_2 \). Los datos de entrada son los hiperparámetros observados \( \theta_i \) en cada conjunto, y la salida es una estimación suave de las densidades \( l(\theta) \) y \( g(\theta) \):


\begin{equation}
	\begin{aligned}
		l(\theta) = \frac{1}{|\mathcal{C}_1| h} \sum_{\theta_i \in \mathcal{C}_1} K\left( \frac{\theta - \theta_i}{h} \right) \\
		g(\theta) = \frac{1}{|\mathcal{C}_2| h} \sum_{\theta_i \in \mathcal{C}_2} K\left( \frac{\theta - \theta_i}{h} \right)
	\end{aligned}
\end{equation}

Al utilizar el método de ventana deslizante de Parzen-Rosenblatt con kernel gaussiano, el TPE obtiene una estimación flexible y no paramétrica de las distribuciones de los hiperparámetros. Esto facilita la exploración eficiente del espacio de búsqueda y la identificación de regiones donde es más probable encontrar hiperparámetros que mejoren el rendimiento del modelo.



	
	
	%%%%%%%%%%%%%%%%%%%%%%%%%%%%%%%%%%%%%%%%%%%%%%
	%% OTRA INFORMACIÓN                         %%
	%%%%%%%%%%%%%%%%%%%%%%%%%%%%%%%%%%%%%%%%%%%%%%
	
	\begin{backmatter}
	
		\section*{Abreviaciones}%% if any
			AD $\rightarrow$ Alzheimer’s Disease (Enfermedad de Alzheimer)\\
			FTD $\rightarrow$ Frontotemporal Dementia (Demencia Frontotemporal)\\
			bvFTD $\rightarrow$ Behavioral variant Frontotemporal Dementia (Variante del Comportamiento de la Demencia Frontotemporal)\\
			PPA $\rightarrow$ Primary Progressive Aphasia (Afasia Progresiva Primaria)\\
			nfvPPA $\rightarrow$ Nonfluent/Agrammatic variant of PPA (Variante no fluente/agramática de la PPA)\\
			svPPA $\rightarrow$ Semantic variant of PPA (Variante semántica de la PPA)\\
			lvPPA $\rightarrow$ Logopenic variant of PPA (Variante logopénica de la PPA)\\
			MND $\rightarrow$ Motor Neuron Disease (Enfermedad de la Motoneurona)\\
			ELA $\rightarrow$ Amyotrophic Lateral Sclerosis (Esclerosis Lateral Amiotrófica)\\
			FTLD $\rightarrow$ Frontotemporal Lobar Degeneration (Degeneración Lobar Frontotemporal)\\
			MRI $\rightarrow$ Magnetic Resonance Imaging (Imagen por Resonancia Magnética)\\
			PET $\rightarrow$ Positron Emission Tomography (Tomografía por Emisión de Positrones)\\
			SPECT $\rightarrow$ Single Photon Emission Computed Tomography (Tomografía por Emisión de Fotón Único)\\
			FDA $\rightarrow$ Food and Drug Administration (Administración de Alimentos y Medicamentos)\\
			GRN $\rightarrow$ Granulin precursor gene (Gen precursor de granulina)\\
			GO $\rightarrow$ Gene Ontology (Gene Ontology) \\
			RQ $\rightarrow$ Research Question (Preguntas de investigación) \\
			SRQ $\rightarrow$ Secondary Research Question (Preguntas de investigación secundaria)			\\
			
			FDR $\rightarrow$ Funtional False Discovery (tasa de descubrimiento falso)
			
			BHO $\rightarrow$ Bayesian Hyperparameter Optimization (ajuste bayesiano de hiperparámetros)	
			
			PDF $\rightarrow$ Probability Density Function (Función de Densidad de Probabilidad)
			
			KDE $\rightarrow$ Kernel Density Estimation (Estimación de la PDF mediante Kernel)
			
		
		\section*{Disponibilidad de datos y materiales}%% if any
			Los datos y materiales se pueden encontrar en el siguiente repositorio de GitHub: https://github.com/MarioPasc/project\_template
		
		\section*{Contribución de los autores}
		
			G.M.A : (Redacción) Análisis Genético del Fenotipo, Introducción; Medidas de Rendimiento, Métodos; Algoritmos, Métodos; Flujo de Trabajo, Diagrama. (Código) Metricas de Rendimiento, Modulo Clustering; Algoritmos de Clústering, Modulo Clustering; Visualización del Frente de Pareto y Visualización Ajuste de Hiperparámetros, Modulo Clustering. 
			
			C.R.G : (Redacción) Relación del Fenotipo con Enfermedades, Introducción; Análisis de enriquecimiento de vías biológicas, Métodos; Red PPI y sus propiedades, Resultados; (Código) Módulo Network Completo; Obtención Proteínas en utils; Cambio de red .tsv a formato iGraph, utils; Creación del fichero original setup.sh y launch.sh; 
			
			A.N.S.B : (Redacción) Definición de Demencia, subtipos, relación con FTD, Introducción; HPO, StringDB, Datos de Entrada, Métodos; Análisis Funcional, Resultados, Discusión; (Código) Módulo Análisis funcional. 
			
			M.P.G : (Redacción) Variantes clínicas de la FTD, Introducción; Objetivos, Objetivos; Optimización, Métodos; (Código) Optimización de los algoritmos, módulo clustering; Visualización de los resultados de clústering, módulo clústering; Revisión continua del código; Modificación continua de launch.sh; Despliegue de la imagen Docker. 
			
			A.P.G. : (Redacción) Tratamientos farmacológicos y terapias, Introducción; Software, paquetes de Python, Métodos; Análisis Funcional, Resultados, Discusión; (Código), Análisis Funcional y filtrado, módulo análisis funcional; Visualización de los términos GO más significativos, módulo de análisis funcional.
	
	
		%%%%%%%%%%%%%%%%%%%%%%%%%%%%%%%%%%%%%%%%%%%%%%%%%%%%%%%%%%%%%%%%%%%%%%%%%%%%%%%%%%%%%%%%
		%% BIBLIOGRAFIA: no teneis que tocar nada, solo sustituir el archivo bibliography.bib %%
		%% por el que hayais generado vosotros                                                %%
		%%%%%%%%%%%%%%%%%%%%%%%%%%%%%%%%%%%%%%%%%%%%%%%%%%%%%%%%%%%%%%%%%%%%%%%%%%%%%%%%%%%%%%%%
		
		\bibliographystyle{bmc-mathphys} % Style BST file (bmc-mathphys, vancouver, spbasic).
		\bibliography{bibliography}      % Bibliography file (usually '*.bib' )
	
	\end{backmatter}
\end{document}
